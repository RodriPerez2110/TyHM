\documentclass[runningheads]{llncs}
\usepackage{graphicx}
\usepackage{xcolor,lipsum}

\begin{document}

\title{Transición Energética}

\author{Agustín Costarelli \and Ramiro Linares \and Rodrigo Perez \and Victor Silva \and Mía Torres López}
%
\authorrunning{Grupo Mecañoñes}
%
\institute{Universidad Nacional de Cuyo, Facultad de Ingeniería, Centro Universitario, Mendoza, Argentina}
%
\maketitle

\section{Movilidad con bajas emisiones de carbono}
% ####################################################################################



\subsection{Análisis del Yacimiento a la Rueda}
La evaluación del ciclo de vida (ACV) se usa normalmente para comparar las emisiones de los combustibles utilizados en los vehículos. El análisis del pozo a la rueda (WTW) es un marco específico de ACV, que sirve para comprender los impactos ambientales del combustible y de los vehículos que lo consumen. La idea general es captar todas las absorciones y emisiones a lo largo de las etapas de producción tanto del combustible como de los vehículos. Los resultados de este tipo de investigaciones, muestran que el combustible en un automóvil, en general, posee los mayores impactos de ciclo de vida. El marco de análisis WTW puede prestarse al comparar combinaciones de diversos vehículos y combustibles. Para comparar los diferentes combustibles, los investigadores y las agencias reguladoras han llevado a cabo análisis de yacimiento a tanque (WTT).

\subsection{Pilas Combustible de Hidrógeno}
El hidrógeno es el elemento más abundante en el universo, constituyendo alrededor del 75\% del total de la materia. En la Tierra, hay mucho hidrógeno unido a otras sustancias químicas, pero es bastante raro encontrar hidrógeno puro. Un atributo clave que hace del hidrógeno una buena opción, es su gran densidad energéticamente, con mucha energía empaquetada en un pequeño volumen (o masa). El hidrógeno tiene una mayor densidad energética que la gasolina. Pero el hidrógeno también es muy volátil y plantea problemas de seguridad por la posibilidad de explosión durante su almacenamiento o transporte. No obstante, la producción de hidrógeno verde a partir de energías renovables podría hacer que las tecnologías de pilas de combustible de hidrógeno formen parte de la flota automovilística en un mundo descarbonizado. Hay varios analistas de la transición energética que han propuesto y promocionado los beneficios de una "\textit{economía del hidrógeno}" (Rifkin 2003). Se están llevando a cabo conversaciones más recientes que tratan sobre el hidrógeno como almacenamiento estacional o en la industria. \\

Definición: \\
Hidróneno verde se refiere comúnmente al hidrógeno producido mediante electrólisis con energía renovable. \\

El hidrógeno es un portador de energía, no una fuente de energía primaria. Como combustible, el hidrógeno puede puede utilizarse de varias formas. Puede ser mezclado con gas natural para su combustión (por ejemplo, para calefacción) o pasar por una pila de hidrógeno para generar energía eléctrica. Las pilas de hidrógeno toman el hidrógeno y oxígeno, y liberan agua y producen energía eléctrica. Son funcionalmente lo contrario de la electrólisis, en la que el agua se divide en hidrógeno y oxígeno, utilizando electricidad. El proceso de electrólisis tiene un índice de eficiencia de entre el 50\% y el 80\%. Los nuevos diseños propuestos para los procesos de electrólisis, utilizan pilas de electrólisis de óxido sólido, las cuales pretenden alcanzar eficiencias del 90-95\%. \\

Definición: \\
Pila de Hidróneno un dispositivo que transforma el hidrogeno y el oxígeno, en agua y electricidad. \\


La obtención del hidrógeno a partir del agua requiere de mucha energía, más de la que se obtiene del propio combustible. La mayor parte del hidrógeno actual procede del reformado del gas natural. También puede producirse mediante la gasificación del carbón y a partir del etanol. El hidrógeno puede obtenerse a partir del metano derivado de la materia vegetal en descomposición producida por la fotosíntesis. El metano del biogás en las instalaciones de tratamiento de aguas residuales es un material primario derivado de la biomasa, aunque puede haber algunos productos de combustibles fósiles en estos flujos de residuos, por ejemplo, plásticos, aceite de motor usado, lubricantes industriales para máquinas, etc. \\

Tomar ideas prestadas de la naturaleza puede ser muy útil. Investigadores de la Universidad de Michigan han desarrollado un dispositivo que utiliza la fotosíntesis artificial. El dispositivo está formado por componentes muy similares a los de la energía fotovoltaica y los LED, como el silicio y el nitruro de galio. El nitruro de galio transforma la luz (en forma de fotones) en electrones y huecos (vacantes de electrones con carga positiva) que fluyen libremente. Cuando los fotones inciden en el dispositivo, el campo eléctrico separa los electrones excitados de los huecos para producir hidrógeno y oxígeno. La fotosíntesis natural tiene una eficacia del 0,6\%, pero este método, según ellos, alcanza el 3\% y, en teoría, podría ser mucho mayor. Dado que el dispositivo, al igual que la fotosíntesis natural, solo utiliza luz solar y agua, podría ser una vía para la producción a gran escala de combustible de hidrógeno limpio. \\

Una de las principales desventajas de las pilas de combustible de hidrógeno para vehículos es su ineficiencia. Tras la producción, la distribución y el almacenamiento, la energía del hidrógeno sigue siendo sólo un tercio de la eficiencia de un vehículo 100\% eléctrico. Sin embargo, a pesar de que un vehículo de hidrógeno puede cargarse significativamente más rápido que la batería de un vehículo eléctrico, 5 minutos frente a hasta 18 horas, hay una falta de infraestructura, como estaciones de carga de hidrógeno, que apoyen el vehículo - las estaciones de carga de los coches eléctricos son mucho más accesibles. Con la falta de infraestructura, será difícil que los consumidores se sientan motivados a comprar los coches Nature. Aunque los coches estén en el mercado, es muy difícil "establecer una base de clientes [fieles], aumentar la producción, [lograr economías de escala] y reducir los costes" (Tollefson 2010). Pero esto ha sido definitivamente un desarrollo positivo para los VE, y es que la gente que los compra los volvería a comprar de nuevo, y la mayoría no volvería a comprar un coche de gasolina. Mientras la revolución de los VE esté liderada por coches seguros, fiables y, con suerte, pronto asequibles, parecen destinados a tener una cuota de mercado cada vez mayor de aquí a 2050. \\

La distribución de una futura automovilidad mediante coches eléctricos con batería, coches eléctricos con pila de hidrógeno e híbridos se basará en el tamaño de los recursos, la disponibilidad y el potencial económico de cada opción, y puede ser regionalmente específica para los combustibles e infraestructuras disponibles. No se trata sólo de la cantidad total de energía (tamaño del recurso) que debe estar disponible, sino también del momento en que esa energía estaría disponible, incluyendo factores como la hora del día, las observaciones de disponibilidad estacional, los kilómetros y la duración de la conducción, y las nuevas ideas sobre cómo deberían utilizarse las carreteras, que podrían no ver un futuro para los automóviles de pasajeros en absoluto, excepto para aquellos que podrían mejorar la accesibilidad.

\subsection{Etanol}
Las plataformas de biocombustibles líquidos basadas en el azúcar y el almidón constituyen la parte más importante del suministro de biocombustibles. Los principales cultivos para la producción de etanol son el maíz y la caña de azúcar, procedentes en gran parte de Estados Unidos y Brasil, respectivamente. Casi un tercio del suministro de maíz estadounidense se destinó a la producción de etanol en 2016 (USDA 2017). \\

Definición: \\
Contenido energético de determinados combustibles líquidos (en kilo-unidades térmicas británicas (kBtu))\\
Gasoline~~~~~~~~~~~~~~~~~~~~~~~~~125 k Btu/gallon\\
Ethanol~~~~~~~~~~~~~~~~~~~~~~~~~~84 k Btu/gallon\\
Compressed natural gas~~~~~106 k Btu/gallon\\
Propane~~~~~~~~~~~~~~~~~~~~~~~~~~91 k Btu/gallon\\

Para utilizar el etanol, podemos utilizar motores de combustible flexible, un motor de combustión interna capaz de aceptar porcentajes más altos de etanol, pero también combustibles líquidos con altas proporciones de gasolina.  La nomenclatura que relaciona estos altos procentajes de gasolina con los de etanol son E10 (10\% de etanol), E100 (todo etanol), E90 (90\% de etanol), etc. \\

Muchos estudios sobre las emisiones de gases de efecto invernadero del etanol sugerían que el balance energético no arrojaba rendimientos positivos. En otras palabras, tenía un retorno energético de inversión (EROI) inferior a 1. Los mejores EROI del etanol de maíz están en el rango de 2 y 3. Las plantas de etanol con peor rendimiento para la descarbonización son las que utilizan carbón para el proceso de calentamiento y tienen grandes cantidades de carbón en la red eléctrica. \\

Si los cultivos energéticos siguen siendo las principales fuentes de combustibles líquidos, competirán con las tierras de alta calidad para la agricultura, creando competiciones entre la producción de alimentos y la de combustibles, u otros usos de la tierra. Durante la rápida expansión del etanol de maíz tras el mandato del etanol, se produjeron los llamados "disturbios de la tortilla" en México, no por la escasez de maíz, sino por el rápido aumento de los precios (McMichael 2009). Esta situación particular puede haber sido una situación históricamente poco crelevante, ya que había varios factores que impulsaban el precio de los alimentos. El telón de fondo fue la crisis financiera de 2007-2008, cuando muchos inversores retiraron su dinero de los productos financieros, como los valores respaldados por hipotecas, y lo trasladaron a las materias primas; las personas que tenían fondos de cobertura y los especuladores buscaban apuestas más seguras, aunque los rendimientos fueran más modestos. Además, las principales empresas de alimentos procesados comenzaron a acaparar cualquier alimento disponible en el mercado para asegurarse de seguir siendo rentables. Fue una tormenta perfecta si se tiene en cuenta el aumento de la demanda de etanol. \\













% ####################################################################################
\end{document}